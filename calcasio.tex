\chapter{Casio}
\section{FX-991ES PLUS}\index{Casio!FX-991ES PLUS}
Calcolatrice prodotta in Cina dalla Casio~\cite{Wikipedia2021}\index{Casio!FX82MS}
\subsection{Test interno}
Il test\[\arcsin(\arccos(\arctan(\tan(\cos(\sin(9))))))=9\] ha come risulta dalla tabella~\vref{tab:CasioFX991ESPLUS}. 
\subsection{PCB}
Sigla PCB: PWB-GY450AX-CL RJA539188 001V01
\subsection{Diagnostic mode}
Premere contemporaneamente i tasti \tastoshift\tastoON\tasto{7}.

Premere 5 volte il tasto \tastoshift

Premere tasto \tastoAC per entrare in contrasto. Modificare il contrasto usare i tasti freccia \tastoFrecciaSinistra e \tastoFrecciaDestro
\subsection{Rom}
GY455X VerE

\begin{table}
	\centering
	\begin{tabular}{lll}
		\toprule
		\multicolumn{1}{c}{Modello}&\multicolumn{1}{c}{Risultato}&\multicolumn{1}{c}{Errore}\\
		\midrule
		Casio FX-991ES PLUS&\num{9.000000007}&\num{7.33338e-9}\\
		\bottomrule
	\end{tabular} 
	\caption{Casio FX-991ES PLUS}
	\label{tab:CasioFX991ESPLUS}
\end{table}